\chapter{示例}

% 页眉
\thispagestyle{fancy}


% 开始内容

\section{文字段落}


\lipsum[1]\footnote{这是一段随机生成的文字。}

\subsection{有序列表}

\begin{enumerate}
   \item 1
   \item 2
   \item 3
\end{enumerate}

\subsection{无序列表}

\begin{itemize}
   \item 1
   \item 2
   \item 3
\end{itemize}

% \lipsum[1]

\section{图片}

引用\cref{fig:1}.

\begin{figure}[htp]
   \centering
   \includegraphics[width=7cm]{pku.pdf}
   \caption{图片示例。}
   \label{fig:1}
\end{figure}

\begin{figure}%
   \centering
   \subfigure[子图1]{\label{fig:a}\includegraphics[width=.45\textwidth]{pku.pdf}}\quad
   \subfigure[子图2]{\label{fig:b}\includegraphics[width=.45\textwidth]{pku.pdf}}
   \caption{多图。}
   \label{fig:2}
\end{figure}



\section{表格}


\begin{table}[htp]
   \centering
   \caption{表格示例.}
   \begin{tabular}{cc}
      \toprule  1 & 2   \\
      \midrule
      内容1         & 内容2 \\
      \bottomrule
   \end{tabular}
   \label{tab:1}
\end{table}

\section{公式}


\begin{equation}
   \bm{F} = m\bm{a}.
   \label{eq:1}
\end{equation}
\cref{eq:1}用到了\texttt{bm}包,可以方便地加粗符号。

\begin{equation}
   \int^b_a f(x) \dif x=F(x)\bigg|^b_a.
   \label{eq:2}
\end{equation}
\cref{eq:2}用到了\texttt{commath}包,可以方便地写微分算符。

\begin{equation}
   \me ^{\pi \mi} + 1 = 0.
   \label{eq:3}
\end{equation}
\cref{eq:3}用到了自定义的 \verb"\me, \mi" 来表示常数$\me,\ \mi$.

\section{代码}

可以设置不同语言来高量代码。

\begin{lstlisting}[language=python]
import numpy as np
\end{lstlisting}

\begin{lstlisting}[language=c]
#include <stdio.h>
main(){
	printf("Hello World");
}
\end{lstlisting}

\section{参考文献}

引用参考文献\upcite{xie2019artificial2}.
