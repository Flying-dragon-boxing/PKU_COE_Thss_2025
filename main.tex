\documentclass[UTF8,oneside,a4,12pt]{ctexbook}
% 各种设置都在下面的文件里
% \usepackage{natbib}
\usepackage{url}
\usepackage{amsmath}
\usepackage{graphicx}
\usepackage{parskip}
\usepackage{adjustbox}
\usepackage{subfigure}  %插入多图时用子图显示的宏包
\usepackage{fancyhdr}
\usepackage{commath}%定义d
% \usepackage[UTF8,heading=true]{ctex}
\usepackage{geometry}
\usepackage{bm}
\usepackage{titlesec}
\usepackage{caption}
\usepackage{paralist}
\usepackage{multirow}
\usepackage{booktabs} % To thicken table lines
\usepackage{titletoc}
\usepackage{diagbox}
\usepackage{bm}
\usepackage{autobreak}
\usepackage{authblk}
\usepackage{indentfirst}
\usepackage{float}
\usepackage{amsthm}
\usepackage{fontspec}
\usepackage{color}
%\usepackage{txfonts} %设置字体为times new roman
\usepackage{lettrine}
\usepackage{nameref}
%\usepackage[nottoc]{tocbibind}
\usepackage{amssymb}%font
\usepackage{lipsum}%make test words
\usepackage{picinpar}%words around the picture
\usepackage[all]{xy}%draw arrow
\usepackage{asymptote}%draw picture
\usepackage[perpage]{footmisc}%脚注每页清零
\usepackage[cmyk]{xcolor}
\usepackage{titling}
\usepackage{lipsum}
\usepackage{setspace}

% 页面尺寸设置
\geometry{
   textwidth=138mm,
   textheight=215mm,
   a4paper,
   left=25mm,
   right=20mm,
   top=25mm,
   bottom=25mm,
   headheight=2cm,
   headsep=7mm,
   footskip=7mm,
   % footnotesep=20mm,
   heightrounded,
}

% \geometry{bottom=2.5cm,left=2cm,right=2cm,top=2.5cm}
\newcommand{\crefrangeconjunction}{ - }
\setlength{\parindent}{2em}

\ctexset{today=big}%日期类型设置

\renewcommand{\chaptermark}[1]{\markboth{\CTEXthechapter\ \ \ #1}{}}

\titlecontents{chapter}[-3pt]{\addvspace{5pt}\filright\sf\bfseries\normalsize}%
{\contentspush{\thecontentslabel \ \ \,}}%
{}{{\mdseries\titlerule*[1pc]{.}}\contentspage}

\titlecontents{section}[12pt]{\addvspace{5pt}\normalsize}%
{\contentspush{\thecontentslabel \ \ \,}}%
{}{\titlerule*[1pc]{.}\contentspage}

\titlecontents{subsection}[24pt]{\addvspace{5pt}\normalsize}%
{\contentspush{\thecontentslabel \ \ \,}}%
{}{\titlerule*[1pc]{.}\contentspage}

\titleformat{\section}
{\normalfont\sanhao}{\thesection}{1em}{}

\titleformat{\chapter}{\vspace{-30pt}\sanhao\sf\bfseries\centering}{第\,\chinese{chapter}\,章}{1em}{}

\titlespacing{\chapter}{0pt}{24pt}{18pt}

\titlespacing{\section}{0pt}{24pt}{18pt}

\titlespacing{\subsection}{0pt}{12pt}{6pt}

\titleformat{\subsection}
{\normalfont\sihao}{\thesubsection}{1em}{}

% ======================================
% = Color de la Universidad de Sevilla =
% ======================================
\usepackage{tikz}
\definecolor{PKUred}{cmyk}{0,1,1,0.45}
%超链接设置(包括颜色)
\usepackage[breaklinks,colorlinks,linkcolor=PKUred,citecolor=PKUred,urlcolor=black]{hyperref}
\usepackage{cleveref}

% 参考文献引用样式设置
\newcommand{\upcite}[1]{\textsuperscript{\textsuperscript{\cite{#1}}}}

% 脚注设置
% \renewcommand*\footnoterule{%
%     \vspace*{-3pt}%
%     {\color{PKUred}\hrule width 2in height 0.4pt}%
%     \vspace*{2.6pt}%
% }


%% Color the bullets of the itemize environment and make the symbol of the third
%% level a diamond instead of an asterisk.
%h\renewcommand*\textbullet{\dag}
\renewcommand*\labelitemi{\color{PKUred}\textbullet}
\renewcommand*\labelitemii{\color{PKUred}--}
\renewcommand*\labelitemiii{\color{PKUred}$\diamond$}
\renewcommand*\labelitemiv{\color{PKUred}\textperiodcentered}


% %%% Equation and float numbering
% \numberwithin{equation}{section}		% Equationnumbering: section.eq#
% \numberwithin{figure}{section}			% Figurenumbering: section.fig#
% \numberwithin{table}{section}				% Tablenumbering: section.tab#


%代码设置
\usepackage{listings}
\usepackage{fontspec} % 定制字体
% \newfontfamily\menlo{Menlo}
\usepackage{xcolor} % 定制颜色
\definecolor{mygreen}{rgb}{0,0.6,0}
\definecolor{mygray}{rgb}{0.5,0.5,0.5}
\definecolor{mymauve}{rgb}{0.58,0,0.82}
\lstset{ %
   backgroundcolor=\color{white},      % choose the background color
   basicstyle=\footnotesize\ttfamily,  % size of fonts used for the code
   columns=fullflexible,
   tabsize=4,
   breaklines=true,               % automatic line breaking only at whitespace
   captionpos=b,                  % sets the caption-position to bottom
   commentstyle=\color{mygreen},  % comment style
   escapeinside={\%*}{*)},        % if you want to add LaTeX within your code
   keywordstyle=\color{blue},     % keyword style
   stringstyle=\color{mymauve}\ttfamily,  % string literal style
   frame=single,
   rulesepcolor=\color{red!20!green!20!blue!20},
   % identifierstyle=\color{red},
   language=c++,
   xleftmargin=4em,xrightmargin=2em, aboveskip=1em,
   framexleftmargin=2em,
   numbers=left
}

%脚注
\renewcommand\thefootnote{\fnsymbol{footnote}}

% 定义常数i、e、积分符号d
\newcommand\mi{\mathrm{i}}
\newcommand\me{\mathrm{e}}

%%% Maketitle metadata
\newcommand{\horrule}[1]{\rule{\linewidth}{#1}}		% Horizontal rule
\newcommand{\tabincell}[2]{\begin{tabular}{@{}#1@{}}#2\end{tabular}}

\newcommand{\chuhao}{\fontsize{42pt}{\baselineskip}\selectfont}
\newcommand{\xiaochuhao}{\fontsize{36pt}{\baselineskip}\selectfont}
\newcommand{\yihao}{\fontsize{26pt}{\baselineskip}\selectfont}
\newcommand{\erhao}{\fontsize{22pt}{\baselineskip}\selectfont}
\newcommand{\xiaoerhao}{\fontsize{18pt}{\baselineskip}\selectfont}
\newcommand{\sanhao}{\fontsize{16pt}{\baselineskip}\selectfont}
\newcommand{\sihao}{\fontsize{14pt}{\baselineskip}\selectfont}
\newcommand{\xiaosihao}{\fontsize{12pt}{\baselineskip}\selectfont}
\newcommand{\wuhao}{\fontsize{10.5pt}{\baselineskip}\selectfont}
\newcommand{\xiaowuhao}{\fontsize{9pt}{\baselineskip}\selectfont}
\newcommand{\liuhao}{\fontsize{7.5pt}{\baselineskip}\selectfont}
\newcommand{\qihao}{\fontsize{5.5pt}{\baselineskip}\selectfont}

% 全角符号变为半角
\catcode`\。=\active
\catcode`\,=\active
\catcode`\;=\active
\catcode`\:=\active
\newcommand{。}{.}
\newcommand{,}{,}
\newcommand{;}{;}
\newcommand{:}{:}

\graphicspath{{fig/}} % 图片的文件路径

% pdf 文件设置
\hypersetup{
   pdfauthor={作者},
   pdftitle={pdf名称}
}

% 在这输入你的标题,会加入页眉
\newcommand\TheTitle{\sf 标题}

% Mac 字体设置
% \setmainfont{TimesNewRomanPSMT}
% \setsansfont{Helvetica-Light}
% \setCJKmainfont[ItalicFont=STKaitiSC-Regular,BoldFont=STSongti-SC-Black]{STSongti-SC-Regular}
% \setCJKsansfont[BoldFont=STHeitiSC-Medium]{STHeitiSC-Light}
% \setCJKmonofont{STKaitiSC-Bold}% 加粗楷体
% \newfontfamily\ktb{STKaitiSC-Bold}

% Win 字体设置
% \setCJKsansfont{SimHei}
% 教务要求黑体必须是SimHei,还请Mac用户下载SimHei字体

% 封面请用 word 转 PDF 后插入

\begin{document}

% 设置图片、表格开头的中文显示
%%%%%%%%%%%%%%%%%%%%%%%%%%%%%%%%%%%%%%%%%%%%%%
\captionsetup[figure]{name={图},labelsep=period}
\captionsetup[table]{name={表},labelsep=period}
\renewcommand\contentsname{\centerline{\xiaoerhao\textsf{目\quad 录}}}
\renewcommand\listfigurename{插图目录}
\renewcommand\listtablename{表格目录}
\renewcommand\refname{参考文献}
\renewcommand\indexname{索引}
\renewcommand\figurename{图}
\renewcommand\tablename{表}
\renewcommand\abstractname{摘\quad 要}
\renewcommand\partname{部分}
\renewcommand\appendixname{附录}
\def\equationautorefname{式}%
\def\footnoteautorefname{脚注}%
\def\itemautorefname{项}%
\def\figureautorefname{图}%
\def\tableautorefname{表}%
\def\partautorefname{篇}%
\def\appendixautorefname{附录}%
\def\chapterautorefname{章}%
\def\sectionautorefname{节}%
\def\subsectionautorefname{小小节}%
\def\subsubsectionautorefname{subsubsection}%
\def\paragraphautorefname{段落}%
\def\subparagraphautorefname{子段落}%
\def\FancyVerbLineautorefname{行}%
\def\theoremautorefname{定理}%
\crefname{figure}{图}{图}
\crefname{equation}{式}{式}
\crefname{table}{表}{表}




\setlength{\baselineskip}{20pt} % 行间距

\begin{center}
   \thispagestyle{empty}
   \fontsize{18pt}{\baselineskip}\bf\textsf{版权声明}
   \vspace{10pt}

\end{center}

任何收存和保管本论文各种版本的单位和个人,未经本论文作者同意,不得将本论文转借他人,亦不得随意复制、抄录、拍照或以任何方式传播。否则,引起有碍作者著作权之问题,将可能承担法律责任。

\newpage

\begin{center}
    \thispagestyle{empty}
    \fontsize{18pt}{\baselineskip}\bf\textsf{摘\quad 要}
    \vspace{10pt}
    
\end{center}

\vspace{10pt}

中文摘要

\begin{figure}[b]
    \textbf{关键词:}关键词1,关键词2
\end{figure}

\newpage

\begin{center}
    \thispagestyle{empty}
    \fontsize{18pt}{\baselineskip}\sf\textbf{ABSTRACT}
    \vspace{10pt}
    
\end{center}

\vspace{10pt}
\setlength{\parindent}{0em}

English abstract.


\begin{figure}[b]
    \textbf{KEY WORDS:} [Keyword 1,\quad Keyword 2,\quad Keyword 3]
\end{figure}

\setlength{\parindent}{2em}

{
   \hypersetup{linkcolor=black}
   \vspace{100pt}
   \newpage
   \pagenumbering{gobble}
   {\centering{\tableofcontents\thispagestyle{empty}}}
   \clearpage
   \pagenumbering{arabic}
   \setcounter{page}{1}
}

% 页眉页脚设置
\fancypagestyle{fancynohead}{
    \fancyhead{}
    \renewcommand\headrulewidth{0pt}
    \fancyfoot[L]{}
    \fancyfoot[C]{\sf \wuhao 第\thepage 页}
    \fancyfoot[R]{}
}

\pagestyle{fancynohead}
\ctexset{chapter = {pagestyle = fancynohead}}


\chapter{引言}
\setcounter{page}{1}
\thispagestyle{fancy}

献给北京大学工学院所有本科生。

\chapter{示例}

% 页面格式
\thispagestyle{fancy}

% 开始内容

\section{文字段落}


\lipsum[1]\footnote{这是一段随机生成的文字。}

\subsection{有序列表}

\begin{enumerate}
    \item 1
    \item 2
    \item 3
\end{enumerate}

\subsection{无序列表}

\begin{itemize}
    \item 1
    \item 2
    \item 3
\end{itemize}

% \lipsum[1]

\section{图片}

引用\cref{fig:1}.

\begin{figure}[htp]
    \centering
    \includegraphics[width=7cm]{pku.pdf}
    \caption{图片示例。}
    \label{fig:1}
\end{figure}

\section{表格}


\begin{table}[htp]
    \centering
    \caption{表格示例.}
    \begin{tabular}{cc}
        \toprule  1 & 2     \\
        \midrule
        内容1       & 内容2 \\
        \bottomrule
    \end{tabular}
    \label{tab:1}
\end{table}

\section{公式}


\begin{equation}
    \bm{F} = m\bm{a}.
    \label{eq:1}
\end{equation}
\cref{eq:1}用到了\texttt{bm}包,可以方便地加粗符号。可以方便地加粗符号。可以方便地加粗符号。可以方便地加粗符号。可以方便地加粗符号。可以方便地加粗符号。可以方便地加粗符号。可以方便地加粗符号。可以方便地加粗符号。可以方便地加粗符号。可以方便地加粗符号。

\begin{equation}
    \int^b_a f(x) \dif x=F(x)\bigg|^b_a.
    \label{eq:2}
\end{equation}
\cref{eq:2}用到了\texttt{commath}包,可以方便地写微分算符。

\begin{equation}
    \me ^{\pi \mi} + 1 = 0.
    \label{eq:3}
\end{equation}
\cref{eq:3}用到了自定义的\verb"\me, \mi"来表示常数$\me,\ \mi$.

\section{代码}

可以设置不同语言来高量代码。

\begin{lstlisting}[language=python]
import numpy as np
\end{lstlisting}

\begin{lstlisting}[language=c]
#include <stdio.h>
main(){
    printf("Hello World");
}
\end{lstlisting}

\section{参考文献}

引用参考文献\cite{xie2019artificial2}.本模版设置了参考文献返回的链接,即参考文献最后的数字。



% 参考文献
\newpage
\bibliographystyle{plain}
\clearpage

% \phantomsection
\addcontentsline{toc}{chapter}{参考文献} %向目录中添加条目,以章的名义
\bibliography{thesis}


% 附录
% \newpage
% \clearpage
% \phantomsection

% \addcontentsline{toc}{chapter}{附录 A} %向目录中添加条目,以章的名义

\chapter*{附录A}


% 开始正文


% 致谢
\newpage
\clearpage
\phantomsection

\addcontentsline{toc}{chapter}{致谢} %向目录中添加条目,以章的名义

\chapter*{致谢}
% 页面格式
\thispagestyle{fancy}

% 开始内容

致谢


% 声明
\newpage
\clearpage
\phantomsection
\thispagestyle{empty}

\addcontentsline{toc}{chapter}{北京大学学位论文原创性声明和使用授权说明} %向目录中添加条目,以章的名义

\chapter*{北京大学学位论文原创性声明和使用授权说明}



\begin{center}
    \vspace{-20pt}
    \sihao\bfseries\centering
原创性声明
\vspace{10pt}
\end{center}

本人郑重声明:所呈交的学位论文,是本人在导师的指导下,独立进行研究工作所取得的成果。除文中已经注明引用的内容外,本论文不含任何其他个人或集体已经发表或撰写过的作品或成果。对本文的研究做出重要贡献的个人和集体,均已在文中以明确方式标明。本声明的法律结果由本人承担。

\vspace{30pt}

\hspace{270pt} 论文作者签名: 

\hspace{270pt} 日期:\qquad \quad   年 \qquad  月 \qquad   日

\begin{center}
    \vspace{80pt}
    \sihao\bfseries\centering
    学位论文使用授权说明
\vspace{10pt}
\end{center}



本人完全了解北京大学关于收集、保存、使用学位论文的规定,即:
\begin{itemize}
    \item 按照学校要求提交学位论文的印刷本和电子版本;
    \item 学校有权保存学位论文的印刷本和电子版,并提供目录检索与阅览服务,在校园网上提供服务;
    \item 学校可以采用影印、缩印、数字化或其它复制手段保存论文;
\end{itemize}

\vspace{30pt}

\hspace{270pt} 论文作者签名: 

\hspace{270pt} 日期:\qquad \quad   年 \qquad  月 \qquad   日


\nocite{*}

\end{document}


